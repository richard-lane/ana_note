\documentclass[note.tex]{subfiles}

\begin{document}

\section{Charm Fitter}
\subsection{Parameter Counting}
\label{Parameter Counting}
Eqs \ref{Gamma eqn minus} and \ref{Gamma eqn plus} are used to measure $\gamma$.
Taking $\Gamma_D$, $x$ and $y$ from external input, we find there are three unknown parameters that are the same in all regions of phase space ($\gamma$, $r_B$ and $\delta_B$) and three parameters that are different between regions ($r_{D, \Omega}$ and two real components of $Z_\Omega^f$).
We therefore have $3N + 3$ unknowns for $N$ regions of phase space.
Fitting to Eq. \ref{amplitudeRatio} in each region gives us $2N$ constraints; Eqs. \ref{Gamma eqn minus} and \ref{Gamma eqn plus} provide a further $2N$ constraints.
We therefore have $4N$ constraints for $3N + 3$ unknowns; we need at least $N\geq 3$ phase space bins to extract all parameters from data.

In principle, any binning scheme can be used to divide the final state phase space into regions.
However, recent developments in $D\rightarrow K \pi \pi \pi$ amplitude models[TODO] enable us to make a model-informed binning in order to make best use of the data.
By binning the data according to the phase difference between the DCS and CF amplitude models at each point,
we find higher coherence in each bin than if we were integrating over the phase space.
This improves the sensitivity to $\gamma$; using this binning scheme is expected to give a precision as low as 5\degree[TODO].

\subsection{Fit Strategy}
The charm interference parameter $Z_\Omega^f$ must be extracted from Eq. \ref{rateRatio} in order to measure $\gamma$.
Practically, this is done by measuring the number of wrong sign (``WS''; Eq.~\ref{approxDRate}) and right sign (``RS''; Eq.~\ref{approxDBarRate})
events in bins of decay time and finding the parameters that best fit the ratio of counts.
A binned $\chi^2$ fit is used for this analysis. The $\chi^2$ is defined in Eq.~\ref{Unconstrained Chi2}.

\begin{equation}
    \chi^2_0 = \sum_{bins}\left[\frac{R^{exp}_{bin} - R^{meas}_{bin}}{\sigma_{bin}}\right]^2
    \label{Unconstrained Chi2}
\end{equation}

Where:
\begin{equation}
    R^{exp}_{bin} = \frac{\int_{bin}\Gamma^{WS}(t; r_D, x, y, Z_\Omega^f, \Gamma_D)dt}{\int_{bin}\Gamma^{RS}(t; r_D, x, y, Z_\Omega^f, \Gamma_D)dt}
    \label{Expected Bin Population}
\end{equation}

As mentioned in Section~\ref{Parameter Counting}, we can take $x$ and $y$ from external input.
This can be done by modifying the $\chi^2$ as in Eq.~\ref{Constrained Chi2}.

\begin{equation}
    \chi^2_{\mathrm{constrained}} = \chi^2_0 + \frac{1}{1 - \sigma_{xy}^2}\left(\frac{(x-x_0)^2}{\sigma_x^2}+\frac{(y-y_0)^2}{\sigma_y^2}-\frac{2\sigma_{xy}(x-x_0)(y-y_0)}{\sigma_x\sigma_y}\right)
    \label{Constrained Chi2}
\end{equation}

Where $\sigma_x$, $\sigma_y$ and $\sigma_xy$ represent the uncertainties and correlations on the world average
measurements of $x$ and $y$, denoted $x_0$ and $y_0$.

It is also possible introduce the constraint on $Z_\Omega^f$ from charm threshhold data by modifying the $\chi^2$.
This is done as follows for the CLEO data by using the CLEO likelihood $L^\mathrm{CLEO}$; Eq.~\ref{CLEO Likelihood}.

\begin{equation}
    \chi^2_{\mathrm{CLEO}} = \chi^2_{\mathrm{constrained}} - 2 L^\mathrm{CLEO}(r_D, x, y, Z_\Omega^f)
    \label{CLEO Likelihood}
\end{equation}

A similar treatment can be used to introduce the constraint from BES-III.

An example of a fit to Eq.~\ref{rateRatio} in time bins is seen in Fig.~\ref{exampleFit}.
\begin{figure}[htb!]
    \centering
    \includegraphics[width=0.75\textwidth]{img/exampleFit.png}
    \caption{An example of the fit to Eq. \ref{rateRatio}.
        Toy decay times were generated according to Eqs. \ref{approxDRate} and \ref{approxDBarRate} and a fit was performed.
        Horizontal error bars show the bins in decay time used; vertical error bars show the error in the ratio.
        Parameters $r_D$ and $Re(Z_\Omega^f)$ were allowed to float in this fit; $x$ and $y$ were constrained using their known[TODO] world averages and uncertainties,
        and $\Gamma_D$ and $Im(Z_\Omega^f)$ were fixed to their true values.
    }
    \label{exampleFit}
\end{figure}

\subsection{$Z_\Omega^f$ Scan}
The purpose of this fitter is to find a constraint on the charm interference parameter $Z_\Omega^f$.
There are three free parameters in Eq. \ref{rateRatio} ($r_{D, \Omega}$ and two components of $Z_\Omega^f$);
the fit provides two constraints. We therefore cannot extract a single best-fit value of $Z_\Omega^f$, but can only restrict $Z_\Omega^f$ to a region in the $Im(Z_\Omega^f) - Re(Z_\Omega^f)$ plane.
Examples of $\chi^2$ scans over possible values of $Z_\Omega^f$ using this fitter are seen in Fig. \ref{Z scans}.

\begin{figure}[htb!]
    \begin{subfigure}[htb!]{0.3\linewidth}
        \includegraphics[width=\textwidth]{img/mixing.png}
        \caption{Pure D mixing}
        \label{mixing scan}
    \end{subfigure}
    \begin{subfigure}[htb!]{0.3\linewidth}
        \includegraphics[width=\textwidth]{img/cleo.png}
        \caption{CLEO-c constraint}
    \end{subfigure}
    \begin{subfigure}[htb!]{0.3\linewidth}
        \includegraphics[width=\textwidth]{img/combined.png}
        \caption{Combination}
    \end{subfigure}
    \centering
    \caption{$\chi^2$ scans showing constraints on $Z_\Omega^f$.
        Each coloured band represents $1/2/3\sigma$, i.e. 39/86/99\% probability.
        The fit from pure D mixing (Eq. \ref{rateRatio}) is on the left, the constraint from CLEO-c[TODO] in the middle and their combination on the right.
        2,000,000 events were simulated for the D-mixing input.
        The combination gives a tighter constraint on $Z_\Omega^f$ than either method individually, allowing for a more precise $\gamma$ measurement.}
    \label{Z scans}
\end{figure}


\section{Fitter Validation}
\begin{itemize}
    \item Pull study
    \item Coverage study
\end{itemize}
\end{document}
