\documentclass[note.tex]{subfiles}

\begin{document}

\section{Charm Fitter}
\begin{itemize}
    \item Parameter count
    \item Fit strategy
    \item Talk about 2d scan
    \item Fit example
    \item 2d scan example
\end{itemize}
\subsection{Parameter Counting}
Eqs \ref{Gamma eqn minus} and \ref{Gamma eqn plus} are used to measure $\gamma$.
Taking $\Gamma_D$, $x$ and $y$ from external input, we find there are three unknown parameters that are the same in all regions of phase space ($\gamma$, $r_B$ and $\delta_B$) and three parameters that are different between regions ($r_{D, \Omega}$ and two real components of $Z_\Omega^f$).
We therefore have $3N + 3$ unknowns for $N$ regions of phase space.
Fitting to Eq. \ref{amplitudeRatio} in each region gives us $2N$ constraints; Eqs. \ref{Gamma eqn minus} and \ref{Gamma eqn plus} provide a further $2N$ constraints.
We therefore have $4N$ constraints for $3N + 3$ unknowns; we need at least $N\geq 3$ phase space bins to extract all parameters from data.

In principle, any binning scheme can be used to divide the final state phase space into regions.
However, recent developments in $D\rightarrow K \pi \pi \pi$ amplitude models[TODO] enable us to make a model-informed binning in order to make best use of the data.
By binning the data according to the phase difference between the DCS and CF amplitude models at each point,
we find higher coherence in each bin than if we were integrating over the phase space.
This improves the sensitivity to $\gamma$; using this binning scheme is expected to give a precision as low as 5\degree[TODO].

\subsection{Fit Strategy}
\begin{itemize}
    \item Unconstrained Fit chi2
    \item Constrained fit chi2
    \item CLEO + constrained chi2
\end{itemize}
The charm interference parameter $Z_\Omega^f$ must be extracted from Eq. \ref{rateRatio} in order to measure $\gamma$.
Practically, this is done by measuring the number of wrong sign (``WS''; Eq.~\ref{approxDRate}) and right sign (``RS''; Eq.~\ref{approxDBarRate})
events in bins of decay time and finding the parameters that best fit the ratio of counts.
A binned $\chi^2$ fit is used for this analysis. The $\chi^2$ is defined in Eq.~\ref{Unconstrained Chi2}.

\begin{equation}
    \chi^2 = \sum_{bins}\left[\frac{R^{exp}_{bin} - R^{meas}_{bin}}{\sigma_{bin}}\right]^2
    \label{Unconstrained Chi2}
\end{equation}

Where:
\begin{equation}
    R^{exp}_{bin} = \frac{\int_{bin}\Gamma^{WS}(t; r_D, x, y, Z_\Omega^f, \Gamma_D)dt}{\int_{bin}\Gamma^{RS}(t; r_D, x, y, Z_\Omega^f, \Gamma_D)dt}
    \label{Expected Bin Population}
\end{equation}


\section{Fitter Validation}
\begin{itemize}
    \item Pull study
    \item Coverage study
\end{itemize}
\end{document}
