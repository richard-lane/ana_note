\documentclass[note.tex]{subfiles}

\begin{document}

\section{Efficiency Correction}
\subsection{Formalism}
Fitting to Eq.~\ref{rateRatio} requires us to correct for the decay-time detector efficiency.
To first order we assume efficiency effects will cancel when we take the ratio, but a full correction of the detector efficiency is attempted here [TODO why?].
Performing the analysis in several phase space bins also requires us to correct for any phase-space dependent efficiency.
Since the phase space of a four-body decay is five dimensional, this gives us an efficiency function defined on a six-dimensional space.
The efficiency $\epsilon(\mathbf{\psi}, t)$ is assumed to only be a function of phase space point $\mathbf{\psi}$, and assumed not to depend on the charge of the hadrons.
[TODO This assumption should be validated].

The efficiency is extracted from simulation, using LHCb Monte Carlo (MC) [TODO specify further].
MC was generated for $D^* \rightarrow D^0(\rightarrow K^\pm \pi^\mp \pi^\pm \pi^\mp)\pi$ decays, using AmpGen[TODO] amplitude models to generate the $D\rightarrow K 3\pi$ events.
It is assumed that the only difference between MC and the amplitude model used to generate it is the detector efficiency $\epsilon(\mathbf{\psi}, t)$; Eq.~\ref{efficiency}.
\begin{equation}
    \epsilon(\mathbf{\psi}, t) = \frac{\Gamma^{\mathrm{MC}}(\mathbf{\psi}, t)}{\Gamma^{\mathrm{model}}(\mathbf{\psi}, t)}
    \label{efficiency}
\end{equation}

Because there are four particles in the final state, the full kinematics are described by 16 scalars (four four-vectors).
Conservation laws give eight constraints.
A further three constraints can be obtrained by setting the orientation of the $z$ axis and the $x-z$ plane, leaving us with 5 parameters.
This gives us our 5d final state phase space.
This space can be parameterised in several ways - the parameterisation chosen for performing the reweighting is detailed in [TODO], and uses the following variables:

\begin{equation}
    \label{CM Parameterisation}
    \begin{aligned}
        M(K^\pm\pi^\pm)\quad
        M(\pi^\mp\pi^\mp) \\
        cos(\theta_+)\qquad
        cos(\theta_-)\qquad
        \phi
    \end{aligned}
\end{equation}

[TODO diagram of angles and stuff] These variables are defined in Fig.~[TODO] and ref [TODO].


\subsection{Reweighting Strategy}
\begin{itemize}
    \item Find efficiency by reweighting MC to AmpGen
    \item MC and AmpGen samples
          \begin{itemize}
              \item Want a single reweighter to reweight both RS and WS
              \item Need to have the right amount of each
              \item Use the amplitude models to find out the proportion of RS and WS that were generated
              \item Algebra
          \end{itemize}
    \item Split data into time bins
          \begin{itemize}
              \item Below lowest bin, just throw away because time weight would be too high...
              \item Might be possible to just ignore phsp efficiency for this bin? or do an average or something?
              \item In each time bin - 5d phsp BDT * 1d histogram division
          \end{itemize}
    \item BDT Reweighter in each time bin
    \item Histogram division in each time bin
\end{itemize}
The basic strategy is to train a reweighter to weight the MC to the amplitude model (``AmpGen'') to correct for the detector efficiency.
Since the detection efficiency approaches 0 at low decay times, a threshhold minimum time of time of $0.94\tau_D$ is used - any decays occurring before this time are discarded.
The remaining decays are binned into three time bins, containing approximately equal events in MC.
In each of these time bins, the phase space efficiency is corrected for using a reweighting BDT from the hep-ml[TODO] package.
The decay time efficiency is found using a histogram division.
The overall efficiency is then found by composing the two, as seen in Eq.~\ref{efficiency split}.

\begin{equation}
    \epsilon(\mathbf{\psi}, t) = \epsilon(\mathbf{\psi}) \epsilon(t)
    \label{efficiency split}
\end{equation}

\subsection{Data Sample}
[TODO we've just ignored time here completely - hopefully that's ok]
It is assumed that the efficiency is dependent only on phase space point, and not on the charge of the hadrons.
As such, it is trained on a combination of $Dbar^0 \rightarrow K^+ \pi^- \pi^+ \pi^-$ (``RS'') and $D^0 \rightarrow K^+ \pi^- \pi^+ \pi^-$ (``WS'') events.
It is assumed that the conjugate decays have the same phase space and efficiency structure. [but can we do something to check this, or do the reweighting with both...?].

The relative amount of RS and WS to train on is found as follows.
Denoting the RS and WS amplitudes as $A^{RS}$ and $A^{WS}$ respectively, we can write the distributions of generated and reconstructed as in Eq.~\ref{reco generated}:
\begin{equation}
    \begin{aligned}
        A^{\mathrm{gen}}(\mathbf{\psi}) = \alpha A^{RS}(\mathbf{\psi}) + \beta A^{WS}(\mathbf{\psi})\quad \\
        A^{\mathrm{reco}}(\mathbf{\psi}) = \epsilon(\mathbf{\psi})(\alpha A^{RS}(\mathbf{\psi}) + \beta A^{WS}(\mathbf{\psi}))
    \end{aligned}
    \label{reco generated}
\end{equation}
Where our amplitude models $A$ give us the amplitude at each phase space point.

We don't know the value of the constants $\alpha$ and $\beta$.
Nonetheless, we can find the proportion of $A^{RS}$ and $A^{WS}$ to compare $A^\mathrm{reco}$ to in order to recover the efficiency.
First, we note that it is sufficient to find the efficiency up to a multiplicative constant.
The expression used to evaluate the efficiency is seen in Eq.~\ref{Efficiency estimate}.

\begin{equation}
        \epsilon(\mathbf{\psi}) \propto \frac{\epsilon(\mathbf{\psi})}{\int \epsilon(\mathbf{\psi}) d\mathbf{\psi}} = \frac{\epsilon(\mathbf{\psi})(\alpha A^{RS}(\mathbf{\psi}) + \beta A^{WS}(\mathbf{\psi}))}{\alpha \int \epsilon(\mathbf{\psi}) d\mathbf{\psi}A^{RS}(\mathbf{\psi}) + \beta \int \epsilon(\mathbf{\psi}) d\mathbf{\psi}A^{WS}(\mathbf{\psi})}
    \label{Efficiency estimate}
\end{equation}

The right hand side of Eq.~\ref{Efficiency estimate} equation is evalated as follows.
First, the factors seen in Eq.~\ref{correction factors} are defined.
\begin{equation}
    \begin{aligned}
        C_\alpha = \alpha \int \epsilon(\mathbf{\psi}, t) d\mathbf{\psi} \\
        C_\beta = \beta \int \epsilon(\mathbf{\psi}, t) d\mathbf{\psi}
    \end{aligned}
    \label{correction factors}
\end{equation}

These can be evaluated using a Riemann sum, seen in Eq.~\ref{evaluating correction factors}.
\begin{equation}
    \begin{aligned}
        C_\alpha = \alpha \int \epsilon(\mathbf{\psi}, t) A^{RS}\frac{1}{A^{RS}}d\mathbf{\psi} = \sum_{RS}\frac{1}{A^{RS}} \\
        C_\beta = \beta \int \epsilon(\mathbf{\psi}, t)A^{WS} \frac{1}{A^{WS}} d\mathbf{\psi} = \sum_{WS}\frac{1}{A^{WS}}
    \end{aligned}
    \label{evaluating correction factors}
\end{equation}

Using these factors, we can rewrite Eq.~\ref{Efficiency estimate} as Eq.~\ref{efficiency c factors}:
\begin{equation}
        \epsilon(\mathbf{\psi}) \propto \frac{\epsilon(\mathbf{\psi})}{\int \epsilon(\mathbf{\psi}) d\mathbf{\psi}} = \frac{\epsilon(\mathbf{\psi})(\alpha A^{RS}(\mathbf{\psi}) + \beta A^{WS}(\mathbf{\psi}))}{C_\alpha A^{RS} + C_\beta A^{WS}}
    \label{efficiency c factors}
\end{equation}

This is the expression used when evaluating the efficiency - the numerator is the combination of the  RS + WS MC samples; the denominator is RS+WS AmpGen generated in the proportions given by $C_\alpha$ and $C_\beta$.

\subsection{Validation}
\begin{itemize}
    \item Projections
    \item Alternate projections
    \item Classifiers
    \item Z scatters
\end{itemize}


\end{document}
