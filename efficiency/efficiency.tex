\documentclass[note.tex]{subfiles}

\begin{document}

\section{Efficiency Correction}
\subsection{Formalism}
Fitting to Eq.~\ref{rateRatio} requires us to correct for the decay-time detector efficiency.
To first order we assume efficiency effects will cancel when we take the ratio, but a full correction of the detector efficiency is attempted here [TODO why?].
Performing the analysis in several phase space bins also requires us to correct for any phase-space dependent efficiency.
Since the phase space of a four-body decay is five dimensional, this gives us an efficiency function defined on a six-dimensional space.
The efficiency $\epsilon(\mathbf{\psi}, t)$ is assumed to only be a function of phase space point $\mathbf{\psi}$, and assumed not to depend on the charge of the hadrons.
[TODO This assumption should be validated].

The efficiency is extracted from simulation, using LHCb Monte Carlo (MC) [TODO specify further].
MC was generated for $D^* \rightarrow D^0(\rightarrow K^\pm \pi^\mp \pi^\pm \pi^\mp)\pi$ decays, using AmpGen[TODO] amplitude models to generate the $D\rightarrow K 3\pi$ events.
It is assumed that the only difference between MC and the amplitude model used to generate it is the detector efficiency $\epsilon(\mathbf{psi}, t)$; Eq.~\ref{efficiency}.
\begin{equation}
    \epsilon(\mathbf{\psi}, t) = \frac{\Gamma^{\mathrm{MC}}(\mathbf{\psi}, t)}{\Gamma^{\mathrm{model}}}
    \label{efficiency}
\end{equation}

Because there are four particles in the final state, the full kinematics are described by 16 scalars (four four-vectors).
Conservation laws give eight constraints.
A further three constraints can be obtrained by setting the orientation of the $z$ axis and the $x-z$ plane, leaving us with 5 parameters.
This gives us our 5d final state phase space.
This space can be parameterised in several ways - the parameterisation chosen for performing the reweighting is detailed in [TODO], and uses the following variables:

\begin{equation}
    \label{CM Parameterisation}
    \begin{aligned}
        M(K^\pm\pi^\pm)\quad
        M(\pi^\mp\pi^\mp)\\
        cos(\theta_+)\qquad
        cos(\theta_-)\qquad
        \phi
    \end{aligned}
\end{equation}

[TODO diagram of angles and stuff] These variables are defined in Fig.~[TODO] and ref [TODO].


\section{Reweighting Strategy}
\begin{itemize}
    \item Find efficiency by reweighting MC to AmpGen
    \item MC and AmpGen samples
          \begin{itemize}
              \item Want a single reweighter to reweight both RS and WS
              \item Need to have the right amount of each
              \item Use the amplitude models to find out the proportion of RS and WS that were generated
              \item Algebra
          \end{itemize}
    \item Split data into time bins
          \begin{itemize}
              \item Below lowest bin, just throw away because time weight would be too high...
              \item Might be possible to just ignore phsp efficiency for this bin? or do an average or something?
              \item In each time bin - 5d phsp BDT * 1d histogram division
          \end{itemize}
    \item BDT Reweighter in each time bin
    \item Histogram division in each time bin
\end{itemize}

\section{Validation}
\begin{itemize}
    \item Projections
    \item Alternate projections
    \item Classifiers
    \item Z scatters
\end{itemize}


\end{document}
