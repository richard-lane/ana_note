\documentclass[note.tex]{subfiles}

\begin{document}

\section{Intro}
Measurement of the CKM angle $\gamma$ is a vital goal of the LHCb experiment and of flavour physics in general[TODO],
and is an excellent test of the standard model since the theoretical uncertainty on $\gamma$ measured from $B\rightarrow DK$ decays is small.
The $K\pi\pi\pi$ final state was chosen for this analysis as the ampltiude structure is expected to lead to an especially precise measurement of $\gamma$[TODO].

The $B \rightarrow DK \rightarrow f$ family of decays, where $f$ denotes a common final state,
provide several ways of measuring the CKM phase $\gamma$. In all cases, sensitivity to
$\gamma$ arises from interference between the intermediary $B \rightarrow DK$ decay amplitudes.
This is seen in the Feynman diagrams, Fig~\ref{fig:bucs}.

\begin{figure}[htb!]
    \begin{subfigure}[t]{0.5\linewidth}
        \centering
        \includegraphics[width=\textwidth]{img/bQuark.png}
    \end{subfigure}
    \begin{subfigure}[t]{0.5\linewidth}
        \centering
        \includegraphics[width=\textwidth]{img/bQuarkSuppressed.png}
    \end{subfigure}
    \caption{Feynman diagrams showing $b$ quark transitions for the favoured (left) and suppressed (right) amplitudes $B^-\rightarrow D^0 K^-$ and $B^-\rightarrow \overline{D}^0 K^-$.
        The $b\rightarrow u$ transition is suppressed by a factor of $V_{ub}$, which introduces a relative weak phase $\gamma$.
        The sign of this phase is reversed for the CP-conjugate $B^+$ decay.}
    \label{fig:bucs}
\end{figure}

The ratio of suppressed to favoured $B^\pm \rightarrow DK^\pm$ amplitudes is seen in Eq. \ref{B ratio eqn}.
\begin{equation}
    \label{B ratio eqn}
    \frac{\mathcal{A}^\pm_{supp}}{\mathcal{A}^\pm_{fav}}= r_B{\rm e}^{\delta_B \pm \gamma}
\end{equation}

Since $r_B$ is small ($\sim 0.1$[TODO]), the overall interference effect is enhanced if the final state is chosen such that the $D^0\
\rightarrow f$ amplitude is suppressed and $\overline{D}^0\rightarrow f$ favoured (or vice-versa) (the ADS method[TODO]).

For example, the $B^- \rightarrow (D \rightarrow K \pi \pi \pi) K^-$ decays proceed via the routes seen in Fig. \ref{b mixing}.

\begin{figure}[htb!]
    \begin{subfigure}[htb!]{0.5\linewidth}
        \includegraphics[width=\textwidth]{img/favbMixingDiagram.png}
    \end{subfigure}
    \begin{subfigure}[htb!]{0.5\linewidth}
        \includegraphics[width=\textwidth]{img/bMixingDiagram.png}
    \end{subfigure}
    \caption{Diagram showing the amplitudes contributing to $B^-\rightarrow K^-(K3\pi)$ via $DK^-$.
        The decay to $K^-\pi^+\pi^+\pi^-$ (left) is completely dominated by the favoured amplitudes (thick arrows), making it a good normalisation mode.
        The decay to $K^+\pi^-\pi^-\pi^+$ (right) is sensitive to the weak phase $\gamma$, as both the decays via $D^0K^-$ and $\overline{D}^
0K^-$ contribute.
    }
    \label{b mixing}
\end{figure}



\section{Charm Mixing}
\begin{itemize}
    \item Feynman diagrams - WS and RS
    \includegraphics[width=0.5\textwidth]{img/rightSignD.png}
    \item Mixing diagrams
    \item Amplitudes
    \item Interference parameter
    \item RS and WS decay rates
    \item Ratio
\end{itemize}

\end{document}
