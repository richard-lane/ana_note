\documentclass[note.tex]{subfiles}

\begin{document}

\section{Intro}
Measurement of the CKM angle $\gamma$ is a vital goal of the LHCb experiment and of flavour physics in general[TODO],
and is an excellent test of the standard model since the theoretical uncertainty on $\gamma$ measured from $B\rightarrow DK$ decays is small.
The $K\pi\pi\pi$ final state was chosen for this analysis as the ampltiude structure is expected to lead to an especially precise measurement of $\gamma$[TODO].

The $B \rightarrow DK \rightarrow f$ family of decays, where $f$ denotes a common final state,
provide several ways of measuring the CKM phase $\gamma$. In all cases, sensitivity to
$\gamma$ arises from interference between the intermediary $B \rightarrow DK$ decay amplitudes.
This is seen in the Feynman diagrams, Fig~\ref{fig:bucs}.

\begin{figure}[htb!]
    \begin{subfigure}[t]{0.5\linewidth}
        \centering
        \includegraphics[width=\textwidth]{img/bQuark.png}
    \end{subfigure}
    \begin{subfigure}[t]{0.5\linewidth}
        \centering
        \includegraphics[width=\textwidth]{img/bQuarkSuppressed.png}
    \end{subfigure}
    \caption{Feynman diagrams showing $b$ quark transitions for the favoured (left) and suppressed (right) amplitudes $B^-\rightarrow D^0 K^-$ and $B^-\rightarrow \overline{D}^0 K^-$.
        The $b\rightarrow u$ transition is suppressed by a factor of $V_{ub}$, which introduces a relative weak phase $\gamma$.
        The sign of this phase is reversed for the CP-conjugate $B^+$ decay.}
    \label{fig:bucs}
\end{figure}

The ratio of suppressed to favoured $B^\pm \rightarrow DK^\pm$ amplitudes is seen in Eq. \ref{B ratio eqn}.
\begin{equation}
    \label{B ratio eqn}
    \frac{\mathcal{A}^\pm_{supp}}{\mathcal{A}^\pm_{fav}}= r_B{\rm e}^{\delta_B \pm \gamma}
\end{equation}

Since $r_B$ is small ($\sim 0.1$[TODO]), the overall interference effect is enhanced if the final state is chosen such that the $D^0\
    \rightarrow f$ amplitude is suppressed and $\overline{D}^0\rightarrow f$ favoured (or vice-versa) (the ADS method[TODO]).

For example, the $B^- \rightarrow (D \rightarrow K \pi \pi \pi) K^-$ decays proceed via the routes seen in Fig. \ref{b mixing}.

\begin{figure}[htb!]
    \begin{subfigure}[htb!]{0.5\linewidth}
        \includegraphics[width=\textwidth]{img/favbMixingDiagram.png}
    \end{subfigure}
    \begin{subfigure}[htb!]{0.5\linewidth}
        \includegraphics[width=\textwidth]{img/bMixingDiagram.png}
    \end{subfigure}
    \caption{Diagram showing the amplitudes contributing to $B^-\rightarrow K^-(K3\pi)$ via $DK^-$.
        The decay to $K^-\pi^+\pi^+\pi^-$ (left) is completely dominated by the favoured amplitudes (thick arrows), making it a good normalisation mode.
        The decay to $K^+\pi^-\pi^-\pi^+$ (right) is sensitive to the weak phase $\gamma$, as both the decays via $D^0K^-$ and $\overline{D}^
            0K^-$ contribute.
    }
    \label{b mixing}
\end{figure}



\section{Charm Mixing}
\begin{itemize}
    \item Feynman diagrams - WS and RS
          \includegraphics[width=0.5\textwidth]{img/rightSignD.png}
    \item Mixing diagrams
    \item Amplitudes
    \item Interference parameter
    \item RS and WS decay rates
    \item Ratio
\end{itemize}
D mesons are produced at the LHC as flavour eigenstates $D^0$ or $\overline{D}^0$.
Feynman diagrams for decays to $K^+ \pi^-$ are seen in Fig. \ref{d Feynman} - the $K^+ \pi^- \pi^- \pi^+$ diagrams are similar.

\begin{figure}[htb!]
    \begin{subfigure}[t]{0.5\linewidth}
        \centering
        \includegraphics[width=\textwidth]{img/rightSignD.png}
        \caption{}
        \label{CF}
    \end{subfigure}
    \begin{subfigure}[t]{0.5\linewidth}
        \centering
        \includegraphics[width=\textwidth]{img/wrongSignD.png}
        \caption{}
        \label{DCS}
    \end{subfigure}
    \caption{Two Feynman diagrams showing possible decays $D^0 \rightarrow K^\mp \pi^\pm$.
        The Cabibbo favoured diagram (left) contains interactions between same-generation quarks.
        The right-hand diagram contains two vertices involving off-diagonal CKM matrix elements, so it is known as doubly Cabibbo suppressed.
    }
    \label{d Feynman}
\end{figure}

This analysis also exploits contributions from D mixing.
This is a slow process and the $\overline{D}^0$ decay (Fig. \ref{CF}) is Cabibbo favoured (CF) - the mixing contribution is negligible. However, the direct $D^0$ decay (Fig. \ref{DCS}) is doubly Cabibbo suppressed (DCS), and so $D^0 \rightarrow \overline{D}^0 \rightarrow f$ is a relevant contribution to the overall $D^0\rightarrow f$ rate.
The two contributions to the $D^0$ decay rate are shown schematically in Fig. \ref{d mixing}.
\begin{figure}[htb!]
    \centering
    \includegraphics[width=0.6\textwidth]{img/dMixingDiagram.png}
    \caption{The amplitudes contributing to $D^0\rightarrow K^+\pi^-\pi^-\pi^+$.
        The interference between them is quantified by the charm mixing parameter Eq. \ref{Charm interference definition}.
    }
    \label{d mixing}
\end{figure}

The following notation is used for the $D^0\rightarrow f$ and $\overline{D}^0\rightarrow f$ amplitudes:
\begin{equation}
    \begin{aligned}
        A(\mathbf{p}) \equiv \Braket{f_{\mathbf{p}} | \hat{H} | D^0} \\
        B(\mathbf{p}) \equiv \Braket{f_{\mathbf{p}} | \hat{H} | \overline{D}^0}
    \end{aligned}
\end{equation}
where $\mathbf{p}$ denotes a point in final state phase space.

Interference between the direct (DCS) $D^0\rightarrow f$ and mixing $D^0\rightarrow \overline{D}^0 \rightarrow f$ amplitudes is described by the charm interference parameter (Eq. \ref{Charm interference}):
\begin{equation}
    \label{Charm interference definition}
    Z_\Omega^f = \frac{\int_\Omega A(\mathbf{p})B^*(\mathbf{p}) d\Phi}{N_\Omega}
\end{equation}
where $\Omega$ is a region of phase space and $N_\Omega$ is a normalisation factor such that $0 \leq |Z_\Omega^f| \leq 1$.

By writing the flavour eigenstates in terms of the mass eigenstates and expanding to third order in the (small) mixing parameters $x$ and $y$
, we arrive at the time-dependent rate equations for $D^0$ and $\overline{D}^0$ decays:
\begin{equation}
    \label{approxDRate}
    \Gamma(D^0(t) \rightarrow f) \approx \Biggl[ \mathcal{A}_\Omega^2
        + \mathcal{A}_\Omega \mathcal{B}_\Omega \left( yRe(Z_\Omega^f) + xIm(Z_\Omega^f)\right)(\Gamma_Dt)
        + \mathcal{B}_\Omega^2 \left( \frac{y^2 + x^2}{4} (\Gamma_Dt)^2 \right)
        \Biggr] e^{-\Gamma_Dt}
\end{equation}
\begin{equation}
    \label{approxDBarRate}
    \Gamma(\overline{D}^0(t) \rightarrow f) \approx \mathcal{B}_\Omega^2 e^{-\Gamma_Dt}
\end{equation}
A full derivation can be found in [TODO].
Dividing Eq.~\ref{approxDRate} by Eq.~\ref{approxDBarRate}, we arrive at Eq.~\ref{rateRatio}:
\begin{equation}
    \label{rateRatio}
    \frac{\Gamma(D^0(t) \rightarrow f)}{\Gamma (\overline{D}^0(t) \rightarrow f)}
    \approx  r_{D,\Omega}^2
    + r_{D,\Omega} \left( yRe(Z_\Omega^f) + xIm(Z_\Omega^f)\right)(\Gamma_Dt)
    + \frac{y^2 + x^2}{4} (\Gamma_Dt)^2
\end{equation}
where the ratio of DCS to CF amplitudes is defined as:
\begin{equation}
    \label{amplitudeRatio}
    r_{D,\Omega} \equiv \frac{\mathcal{A}_\Omega}{\mathcal{B}_\Omega}
\end{equation}
Fitting to Eq. \ref{rateRatio} provides a constraint on $Z_\Omega^f$.

Alternatively, $Z_\Omega^f$ can be taken as external input from charm threshold data (CLEO, BES-III).
Previous studies have shown that combining charm threshold data with input from charm mixing (Eq. \ref{rateRatio}) results in vastly improved precision of both $Z_\Omega^f$ and $\gamma$, with a precision of a few degrees [TODO].

\end{document}
